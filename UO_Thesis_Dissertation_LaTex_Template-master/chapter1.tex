% !TEX root =  main.tex
%--- Chapter 1 ----------------------------------------------------------------%
%   
\chapter{Introduction} 
The research behind developing software toolkits for scientific problems has significantly improved along the years with a variety of tools available for efficiently solving high volumes of mathematical computations. Several different optimizations in algorithms have spawned a vast array of toolkits with each offering a unique advantage to specific problem characteristics, but a silver bullet that efficiently performs well across all unique problems and experimental setups remain elusive. This brings about a new selection concern where the wrong choice of an algorithm could lead to massive performance downgrades that adds up with multiple iterations of the experiment performed across an application. 

This is compounded by the fact that the selection is usually made by researchers who are specialists in that application but have limited knowledge when it comes to identifying the minor intricacies in experimental setups that could lead to significant performance improvements.

In this thesis, we provide a model-based approach towards solving this selection concern for two classes of algorithms, namely, selection of parallel graph processing packages and solvers for linear systems. A web-based framework using the proposed approaches was also developed for these applications. It is essential for researchers to choose reliable algorithms and toolkits, and we aim to provide this solution. Reliability refers to not only having consistent performance improvements but also never to have experiments that fail, which in reality is a common concern.

One way of ensuring reliability is to choose algorithms from the empirical analysis of exhaustively running all combinations of experiment setups. But this scales up pretty quickly making it unfeasible to run experiments just for selection. For example, there are over 120 solvers available for linear systems, and each can be used on many different experimental setups. It would take weeks or months to figure out which experimental setup and algorithm work best for all their linear systems in an application. A model-based approach circumvents all this by using machine learning techniques to predict the quality of solvers for specific systems rather than explicitly running those experiments to determine quality.

The modeling is done on the metadata of the problem set, which tries to capture the relation between the problem and the criterion variable. The criterion variable for both our applications is execution time while the metadata acts as features of the learning models. The fundamental motivation behind our approach towards selection is the fact that given a problem set, the time taken to perform feature extraction and modeling for criterion variable is still less than actually running that experiment. 

Chapters two and three deal with background information and related works, while chapter four presents our framework and its workings. In specific, we provide both a predictive and classification modeling based approach towards algorithm selection for the two mentioned classes of algorithm. For the latter class, we have also provided a ranking framework that gives ranked list of solvers for specific linear systems. Chapter four discusses and analyses the results we obtained for all the experiments performed, followed by the conclusion chapter, which gives us a holistic view of the framework and the final models selected for use. Lastly, chapter seven offers details of the current and future work conducted by us along the same lines and identifies where improvements could be made to the existing methodologies.

