\chapter{RELATED WORK}  

Improving quality and performance of software toolkits is a really mature field with a lot of active research going towards development of new algorithms and techniques for efficiently solving various problems. But a lot of these solutions are still application dependent and works with different degrees of performance for each new application. The end performance is also a factor of the experimental setup and the hardware configurations which are not extensively taken into account when developing these toolkits. The combination of these facts have resulted in a plethora of software toolkits available for most problems. But with that said, not a lot of research has gone into selecting the best toolkits for specific problems with the characteristics of application and experimental setup in mind.

In specific, the two classes of algorithms we would be looking at, namely, algorithms for parallel graph processing and solving linear systems are further unexplored when it comes to choosing the right software for varying experimental setups. An extensive survey on large scale graph processing packages is done by \cite{batarfi2015large}, analysing the performance metrics of various graph packages.  Surveys of iterative solvers and preconditioners are relatively more explored with \cite{benzi2002preconditioning} giving us performance evaluations for solvers focusing on large linear systems. Likewise, \cite{bruaset2018survey}, \cite{chow2006survey}, \cite{butrylo2004survey} also gives us a comprehensive survey of solvers and preconditioners.

But it is to be noted that all of the mentioned works are surveys on existing solutions that runs empirical experiments to analyze performance. They provide inferences on best performing solvers based on empirical evidence but none of them suggest solvers for specific problems, let alone experimental setups. Papers \cite{xu2008satzilla} and \cite{malitsky2012parallel} takes one step closer by providing a portfolio based approach for selecting SAT solvers. But these are once again done through empirical evidences. 

The closest work to us that uses machine learning based models for software solver selections are \cite{kadioglu2011algorithm} and \cite{bhowmick2009towards}. The former uses K-nearest neighbor based approach while the the latter is the closest to our research which uses a similar feature selection and classification based approach for selection. But none of them to the best of our knowledge, perform predictive modelling. In this thesis, we look at selection of software toolkits with a machine learning model based approaches that uses both predictive modeling and classification for fast and accurate selection.